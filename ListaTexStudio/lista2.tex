\documentclass[a4paper, 12pt, fleqn]{article}
\usepackage[utf8]{inputenc}
\usepackage[brazilian]{babel}
\usepackage{indentfirst}
\usepackage{url}
\usepackage{graphicx}
\usepackage{titlesec}
\usepackage{float}
\usepackage{subfigure}
\usepackage{tocbibind}%para bibliografia aparecer no sumário
\usepackage{hyperref}
\usepackage[a4paper, margin=1in]{geometry}
\usepackage{scrextend} %para manipular margens no meio do texto
\usepackage{enumitem}%manejo de itemize
%\usepackage{enumerate}
\usepackage{amsmath}


\renewcommand{\rmdefault}{ptm}

\setlength{\parskip}{1em}

\titlelabel{\thetitle.  }
\thispagestyle{empty}

\begin{document}
	\begin{center}
		\textbf{UNIVERSIDADE FEDERAL DO RIO GRANDE DO NORTE\\		
		CENTRO DE TECNOLOGIA\\		
		DEPARTAMENTO DE ENGENHARIA DE COMPUTAÇÃO E AUTOMAÇÃO\\		
		DISCIPLINA DE TÓPICOS EM ENGENHARIA DE COMPUTAÇÃO}
	\end{center}

	\begin{tabular}{rl}
		Alunos:  &Eric Calasans de Barros\\
		&Fagner Ferreira
	\end{tabular}
		

	\begin{center}
		\large{\textbf{LISTA 2}}
	\end{center}
	

	\section{Questão 1}
		Seja...

\end{document}
